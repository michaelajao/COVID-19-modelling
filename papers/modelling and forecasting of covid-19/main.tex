\documentclass{article}


\usepackage{arxiv}

\usepackage[utf8]{inputenc}
\usepackage[T1]{fontenc}
\usepackage{amsmath,amssymb,amsfonts}
\usepackage{hyperref}
\usepackage{url}
\usepackage{booktabs}      % blackboard math symbols
\usepackage{nicefrac}       % compact symbols for 1/2, etc.
\usepackage{microtype}
\usepackage{amsfonts}       % blackboard math symbols
\usepackage{graphicx}
\graphicspath{ {./images/} }

\title{Optimisation Methods for Vaccine allocation during the COVID-19 Pandemic: A literature Review}
\author{
    Michael Ajao-olarinoye\\
    Centre for computational science and mathematical modelling\\
    coventry university\\
    Coventry, CV1 3EQ \\
    \texttt{olarinoyem@uni.coventry.ac.uk}
}

\begin{document}

\maketitle

\begin{abstract}
    The current COVID-19 epidemic has highlighted the need of the effective and equitable distribution of vital medical supplies, such as vaccines. While the rush to produce and deliver vaccinations continues, decision-makers must figure out how to allocate limited supplies to the most vulnerable groups in a timely and efficient manner. This literature review examines the current state of knowledge on optimization methods for vaccine allocation during pandemics. we conducted a systematic rapid review search of Scopus to identify and analyse articles related to vaccine allocation/distribution strategies and optimisation methods. In this review of the literature, we surveyed 22 original peer-reviewed articles out of 213 articles published between 2019 and 2023 in the Scopus database on optimisation methods used to optimise vaccine distribution or allocation during a pandemic. We identified several key algorithmic approaches, including linear programming, integer programming and stocastics optimization. These approaches aim to minimise various objective functions considering factors such as demographics of the population, vaccine availability, and logistical constraints. Our analysis focuses on the solution methods implemented, the vaccine supply chain, and, if possible, epidemiological methods. In general, our review suggests that algorithmic approaches promise to improve pandemic resource allocation and help decision-makers prioritise limited resources equitably and efficiently.
\end{abstract}

\keywords{vaccine allocation \and optimization \and COVID-19 \and epidemiological models \and epidemics}

\section{Introduction}
\label{sec:intro}

The current COVID-19 pandemic has underlined the need for an effective and fair distribution of key medical resources as an intervention to reduce or eradicate the spread of the virus. At the beginning of the pandemic, there were few reliable models and insufficient data on confirmed cases and fatalities, making it impossible to anticipate future cases and prepare for resource requirements and efficient mitigation methods. Algorithmic approaches, which take advantage of the most current breakthroughs in data science and artificial intelligence (AI), have been offered as a potential solution to this problem. An intervention that is regularly investigated and included in the literature on the allocation of medical resources is vaccination. As the race to develop and distribute vaccines continues, decision makers and government agencies are faced with the challenges of who to vaccinate, when to vaccinate, and where to locate vaccination centres \cite{Bertsimas2022179}.

\subsection{Background information on vaccine allocation during pandemics}
\label{sec:background}
\subsection{Importance of optimizing vaccine allocation}
\label{sec:importance}
\subsection{Research question and objectives}
\subsection{Rationale for conducting a literature review}
\subsection{Overview of the paper}


\section{Search methodology}
\label{sec:methodology}
\subsection{Search strategy and selection criteria}
\subsection{Data sources and study selection }
\subsection{Data extraction and analysis}

\section{Results}
\label{sec:results}
\subsection{Summary of the literature search and selection process }
\subsection{Overview of the vaccine allocation strategies identified in the literature}
\subsection{Review of the optimization methods used in vaccine allocation}
\subsection{Evaluation of the effectiveness of different vaccine allocation strategies}

\section{Discussion}
\label{sec:Discussion}
\section{Conclusion}
\label{sec:Conclusion}
\section{Reference}
\bibliographystyle{unsrt}
\bibliography{references}
\end{document}
